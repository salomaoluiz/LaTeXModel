\chapter*{Introdução}
Em geral, disciplinas de Álgebra Linear e Probabilidade fazem-se presentes na estrutura
curricular de vários cursos na área das ciências exatas. A Álgebra Linear, por sua vez, trata de
manipulações realizadas em objetos matemáticos denominados matrizes e vetores. A disciplina
de Probabilidade tanto pode ser denominada com esta nomenclatura, ou seus conceitos podem
ser estudados na disciplina de Estatística.

As duas disciplinas possuem um papel importante nestes cursos de graduação, pois ambas fornecem instrumentos teóricos potenciais para a interpretação e a resolução de diversos
problemas. Embora estas disciplinas estejam dissociadas no currículo, percebemos que a Álgebra Linear oportuniza uma organização de variáveis e garante algumas operações específicas,
de modo que a Probabilidade pode quantificar as iterações entre essas variáveis e fazer predi-
ções a partir destas iterações, sendo de extrema importância para a realização das inferências
estatísticas.

A Álgebra Linear, por sua estrutura axiomática, pode ser de difícil compreensão por parte
dos estudantes. Tais dificuldades têm se configurado como um desafio para professores e alunos
envolvidos neste processo de aprendizagem. Na busca de elementos que propiciem diferentes
significados aos conceitos algébricos, percebemos que os modelos estocásticos poderiam oportunizar uma situação de ensino a fim de contribuir com o desenvolvimento da aprendizagem.
Neste contexto, nossa opção foi inserir a Cadeia de Markov como ferramenta matemática para
(re)construção de conceitos algébricos e operações algébricas.

A Cadeia de Markov é um processo estocástico, no qual os estados se apresentam de modo
discreto e finito. Neste sentido, a cadeia de Markov ocasiona os chamados processos Markovianos, caracterizados no pressuposto de que o conhecimento do estado futuro depende apenas
do estado atual, isto é, estados anteriores são desconsiderados para as predições do próximo
estado. Este processo foi denominado em homenagem ao matemático Andrei Andreyevich
Markov (1856-1922).

Sendo assim, apresentamos uma sequência de ensino alternativa utilizando-se da Cadeia
de Markov como aplicação de conceitos de Álgebra Linear que tem por objetivo tonar-se um
material didático-pedagógico alternativo para professores universitários que ministram tal disciplina. Além disso, aliado a disponibilidade de softwares gratuitos, nossa proposta pôde ser incrementada com as possibilidades que estes recursos tecnológicos educacionais oferecem.
Deste modo, nosso trabalho organiza-se da seguinte forma:

O Capítulo 2 apresenta conceitos e resultados introdutórios de Álgebra Linear. O Capítulo
3 apresenta a cadeia de Markov, com um breve histórico e suas peculiaridades. No Capítulo 4
abordamos situações-problemas que podem ser desenvolvidas no contexto da sala de aula.

Ressaltamos que, muito além da elaboração de um material de estudos, preocupamo-nos
com a relação existente entre os conteúdos ensinados e a formação acadêmica em diversos
aspectos. Sendo assim, acreditamos deixar nossa contribuição na reflexão do professor sobre a
sua prática e na proficiência de conceitos por parte dos alunos