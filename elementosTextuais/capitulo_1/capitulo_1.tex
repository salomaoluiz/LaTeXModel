
\chapter{Softwares Utilizados}

\section{MikTeX}
\section{TeXstudio}
\subsection{Principais Características do TeXstudio}
\chapter{Configurações do Preâmbulo}
\section{Configurações do \textbackslash documentclass}
O \lstinline|\documentclass{}| é uma das configurações básicas de um documento \LaTeX, nele é onde é definido as principais características do documento, como se sera um trabalho acadêmico, um artigo, banner ou slide. Por ele é possível configurar os padrões do documento, como tamanho de fonte ou da folha, a linguagem, configurações de titulo e subtítulos.

As configurações utilizadas no \lstinline|\documentclass{}| foram:
\begin{itemize}
	\item 12pt $\rarrow$ Tamanho de Fonte 12												
	\item openright $\rarrow$ Inicia a pagina pela direita										
	\item twoside $\rarrow$ A pagina sera impressa frente e verso
	\item a4paper $\rarrow$ O papel padrão com tamanho A4									
	\item brazil $\rarrow$ Define a linguagem com Português-Brasil							
	\item abntex2 $\rarrow$ Para utilizar a classe do documento com normas ABNT				
\end{itemize}

